\documentclass{book}
\usepackage{blindtext}
\usepackage[T1]{fontenc}
\usepackage[utf8]{inputenc}
\usepackage{booktabs}
\usepackage{csvsimple}
\usepackage[a4paper,width=150mm,top=25mm,bottom=25mm,bindingoffset=10mm,headsep=10mm]{geometry}
\usepackage{fancyhdr}
\renewcommand{\headrulewidth}{1pt}
\pagestyle{fancy}
\fancyhead[RO,LE]{Chapter \thechapter}

\title{Mundo de Kuara}
\author{Lucas Tatsuya Tanaka}
\date{\today}

\begin{document}
\maketitle
\tableofcontents

\vspace*{\fill} 
\begin{quote} 
\centering 
\Large O mundo esta mudando, e os velhos deuses sao os responsaveis
\end{quote}
\vspace*{\fill}

\tableofcontents
\part{Ideia}
\begin{itemize}
    \item Ideia Central:O mundo esta mudando, e os velhos deuses dragoes sao os responsaveis
    \item Consequencia:A fortificacao do mistico no mundo e a revelacao do mistico antes nao 
        conhecido, o mundo valoriza os quais se parecem com os deuses, os que se aproveitavam 
        da ausencia dos deuses querem tentar manter os seus poderes 
    \item Mergulho: Com o falta dos deuses muitos se agaram a religiosos 
    \item Resumo:
\end{itemize}

\chapter{Nomeclatura}

criar os nomes das capitais com nomes tupi guaranis assim como o nome das coisas relacionadas ao
mundo draconico deichar como se fosse a lingua morta dos dragoes 

\begin{itemize}
      \item  dragao branco que deu inicio a tudo = Kuara = sol em tupi
      \item  Azul conhecimento (vida) =  Azcoda (juncao de palavras)
      \item  Verde Natureza (vida) = Venada (juncao de palavras)
      \item  Amarelo Tempestade (neutro) = Amten (juncao das palavras)
      \item  Vermelho dragao da guerra (conflito) (morte)= Verguemo (juncao de palavras)
      \item  Roxo dragao da enganacao (comunicacao) (morte)= Roenmo (juncao de plavras)
\end{itemize}

todos os deuses dragoes sao possiveis de utilizar os poderes de morte e vida de criar e destruir 
porem eles tem maior afinidade com certos aspectos os deuses azul e verde sao mais proximos com
os aspectos de vida ja os deuses roxo e vermelho sao mais proximos com os aspectos de morte 
ja o deus amarelo nao tem um proximidade com nenhuma das duas 

Eh necessario representar 8 aspectos sociais dentro dos clerigos 
\begin{itemize}
      \item Conhecimento
      \item Enganacao 
      \item Guerra 
      \item Luz 
      \item Morte
      \item Natureza 
      \item Tempestade 
      \item Vida 
\end{itemize}



\part{Mitos}
\chapter{Mitos de Criacao}
No comeco de tudo o mundo era totalmente vazio e desprovido de qualquer forma ou sentido e devido
a essa falta, a cor preta pode surgir e por ela surgir o branco surgiu ao mesmo tempo, da juncao 
de ambos se fez o cinza e da existencia de tais cores o tempo pode se manifestar no preto o 
branco a existencia na forma de um dragao branco e o cinza o equilibrio ou a magia, porem devido
o dragao ser a unica existencia em meio ao nada decide se partilhar, e de sua partilha suas 
partes se misturaram as cores cinzas e pretas e fizeram surgir 5 novas cores nas quais apois 
anos solitarios decidem se entreter piintando agora o universo sao pintados primeiramente 
algumas subdivinidades (nao necessariamente boas ou mas) sao criados os dragoes censientes e os 
draconantes que apos tempos de vivencia solitarias foram criadas novas racas a partir do sangue 
dos dragoes e os corpos dos dracontes, foram criadas e derivadas todos as outras racas junto 

\part{Racas}
\chapter{Racas Jogaveis}
\section{Draconatos}
A raca dos draconatos surge de uma vontade dos deuses em criar um ser que pudesse interagir 
com o mundo onde vivia de uma maneira mais sincera e simples e tiveram como base sua previa 
cricao os primeiros seres, os dragoes o que fazia com que os draconatos se portassem e tivessem as
estilisticas de um dragao e suas diversas caracteristicas.

A raca dos draconatos e dita ser a primeira raca humanoide a vir a existir e des de sua cricao 
ate o termino da terceira era eles eram considerados os unicos seres humanoides a habitar Kuara 
a dita raca mais proxima dos deuses ate aquele momento porem apos  a 3 era, com o surgimento de 
novas racas humanoides a prioridade que os deuses davam 
foi perdida pois dentro dos deuses agora seu foco estavam em outras criacaoes.

Com o comeco da raca as suas primeiras organizacoes eram feitas atravez de pequenas tribos que 
adoravam os deuses e tinham um estilo de vida cedentario devido a ajuda dada de seus criadoresm,
os draconatos e os dragoes sempre tiveram uma ligacao tanto de sangue quanto espiritual 
porem no comeco de sua existencia haviam diversos conflitos entre ambos pela atencao dos deuses
oque fazia ambas as racas nao conseguirem progredir tanto espiritualmente como tecnologicamente, 
com a passagem do tempo a interferencia dos deuses se fez presente, utilizando ambos os lideres 
dos dragoes como dos draconatos apasiguando a relacao de ambos e fazendo a relacao de ambos se 
transformarem, criando os primeiros imperios draconianos 

\subsection*{Modernidade}
Os draconatos na modernidade estao concentrados nas grandes capitais devido ao seu estato como 
raca nobre, em sua maioria devido a historia rica de seu povo e pela sua aparencia sao muito 
ligados a trabalhos formais como lideres politicos, clerigos, guerreiros entre outros, uma 
pequena quantia insignificante esta ligada a qulquer trabalho que envolva o campo, pastoreio ou
caca (Caçada)

\section{Elfo}
A raca dos elfos tem uma antiga ancestralidade com os draconatos e os tiefilings, ambos tiefilings 
e elfos foram criados na 4 era decorridos dos draconatos, no comeco de sua existencia no mundo 
de Kuara estavam primariamente concentrados nas florestas de Caçapava (lugar de atravesar a mata)
uma extensiva floresta a muito nao explorada ate as eras atuais, sua origem ocorre devido ao 
dragao verde Venada que descidiu utilizar de molde a primeira criacao humanoide para ter uma 
raca ao qual podera guiar visto as capacidades dos draconatos e de seres censientes. 

Visto as prosperidades criadas atraves do contato com os dragao Venada que apresentava o 
conhecimento de como se manter uma sociedade acabou por se tornar o guia para esta nova
sociedade que agora nao tinha necessidade de aprender atraves da tentativa e erro como os 
draconatos tiveram junto aos dragoes, fazendo assim uma rapida evolucao e um contato mistico 
forticimo com o dragao Venada e devocao a este.

As primeiras sociedade elficas e se basiavam em um tipo de sociedade clerical na qual havia um 
clerico que encarnava a presenca do deus Venada que acabava por governar de maneira absoluta,

\subsection*{Modernidade}
Os High elfs se concentram na floresta de cacapava os elfos sao mais comuns em sua maioria pelos 
diversos reinos 

\section{Tiefiling}
A raca dos tiefilings tem uma antiga historia que data a 4 era e tem origem junto aos elfos e 
uma ancestraliade ligada a ambos os draconatos e elfos.

Os tiefilings no comeco de sua existencia eram hereditarios do deserto Ybytata (chao de fogo) 
um local com uma grande dificuldade para a sobrevivencia, assim como os elfos eles descendem 
da criacao passada dos deuses, os draconatos.

O dragao Roenmo descidiu criar e ser um guia de uma raca que iria residir no deserto que apesar 
de ser um lugar de dificil vivencia, para as racas nobre nao passavam de meros lugares devido 
a sua resistencia herdada de seu semelhantes dragoes diferente dos outros deuses Roenmo nao 
gostava de ser tao ativamente aparente as suas criacaoes por isso descidiu ele criar um guia 
para os tiefilings atravez de um objeto magico semelhante a um livro que guiaria a sociedade 
tiefiling durante suas peregrinacaoes no deserto e da troca de conhecimentos possiveis atravez
do livro passado a eles

Entre as racas nobres os tiefilings vivem bem menos do que as outras racas nobres devido a uma 
maldicao dada por Roenmo pela raca trair os seguimentos do livro sagrado convocando demonios 
para sua adoracao afim de ter o poder 

\subsection*{Modernidade}

Obs: colocar a cor Roxa como algo importante para para os tiefilings 

\section{Anao}
criando na 5 era atraves de um esforco coletivo dos deuses 
Os primeiros anoes surgiram das montanhas e cavernas e devido ao constante contato com outras 
racas tiveram um comeco na historia bem conturbado des de escravidoes a intensas batalhas 
oque acabou por fazer os anoes terem uma grande comunidade unida que pode se juntar e gerar 
grandes imperios porem com a chegada da metade da 6 era os imperios anoes comecaram a se 
desmanchar e uma grande diminuicao da quantidade populacional dos anoes 

\subsection*{Modernidade}

Obs: colocar que os Anoes apresentavam um grande imperio porem ele foi destruido na 6 era 

\section{Humano}
criados na 5 era atravez de um esforco coletivo dos deuses 
Os primeiros sers humanos tinham uma cultura nomade muito forte pre o contato com outras racas 
e sua rapida socializacao ao mundo politico e economico dos reinos presentes da 5 era

\subsection{Modernidade}

\section{Halfiling}
criado na 5 era 

\subsection{Modernidade}
apresentam comunidades de principalmente fazendeiros e agricultores tendem a ter uma vida pacata
e tendem a nao buscar muito coisas alem disso porem aqueles buscam algo fora desta vida 

\section{Gnomo}
criado na 5 era

\subsection{Modernidade}

\section{Orc}
criado na 5 era 
\subsection{Modernidade}

\chapter{Racas nao jogaveis}
\section{Dragoes}
No comeco da existencia dos mundos e das primeiras pinturas que os deuses fizeram os primeiros 
seres que foram criados eram dragoes pois tinham sua imagem e semelhanca aos deuses assim como
os deuses eram dotados de censiencia 
\subsection*{Modernidade}

\section{Demonios} 
\subsection*{Modernidade}

\part{Organizacao Social}

Dentro do universo de roda do dragao no tempo atual (a era da renovacao) tem como principais 3 
racas Nobres decorridas de uma justificativa religiosa ereditaria magica e de proximidade com os 
deuses e desenvolvimento dos antigos imperios
\section{Racas Nobres}
Draconianos criados na 2 era e tiefilings e elfos na 4 era 
Draconatos,Tiefilins,Elfos,Meias-Racas com Raca Nobre
Obs: ideia de racas nobres so vai sair apos a 5 era com o abondono dos deuses do mundo e agora 
uma tentativa de justificativa dos antigos imperios em uma superioridade para o dominio das 
outras racas 

\section{Racas Baixas}
Criadas na 5 era 
Humanos,Halfilings,Anoes,Gnomos,Meia-racas com racas Baixas

\chapter{Imperios}
Dentro do mundo de Kuara na era de renovacao existem 3 grandes imperios no mundo, o imperio 
Draconiano, Tiefiling e elfico 

\section{Imperio Draconiano}

\subsection*{Historia}
O imperio Draconiano tem suas origens no comeco das eras e no inicio das primeiras organizacoes 
sociais, o contato das primeiras tribos de draconatos com os dragoes sencientes 

- Dragao rei
- Draconatos
- Tiefilings
- Elfos 
- Racas Baixas

\section{Imperia Elfico}
\subsection*{Historia}
- O Grande Conselho Elfico (nobres Elfos)
- O Clericado 
- Os Guerreiros 
- OS cidadaos

\section{Imperio Tiefilin}
\subsection*{Historia}
- Lorde Tiefiling
- Os tiefilings clerigos e guerreiros  
- Racas Nobres e Baixas 

Clerigos na sociedade estao ligados a rituais que involvem geralmente cores para a adoracao de 
cada deus

Obs: 
	 feiticeiro = poder interno
     mago = estudo
     bruxo = patronos, pactos com deuses ou entidaes superiores
     existe uma sociedade dividida em castas na qual os no topo da piramida sao os dragoes em 
     segundo os dracontos nobres
\chapter{Nobreza}
Os titulos presentes em todos os reinos nobres
\section{Titulos gerais}
Os titulos gerais apenas sao reconhecidos pelos imperios das racas nobres
\begin{itemize}
    \item 
    \item
\end{itemize}
\section{Titulos exclusivos dos Imperios Draconatos}

\begin{itemize}
    \item Alto Draco-Imperador
    \item
\end{itemize}

\section{Titulos exclusivos dos imperios Tiefilings}
\begin{itemize}
    \item
    \item
\end{itemize}

\section{Titulos exclusivos dos Imperios Elficos}
\begin{itemize}
    \item
    \item
\end{itemize}

\section{Titulos exclusivos dos Imperios Magos}
\begin{itemize}
    \item
    \item
\end{itemize}

\part{Eras}
No mundo de Kuara existem 10 eras cada uma com suas particularidades

\chapter{1 era}
A criacao do mundo  e dos primeiros seres
\chapter{2 era}
O surgimendo dos primeiros draconatos 
\chapter{3 era}
\chapter{4 era}
O surgimento de elfos e tiefilings
\chapter{5 era (A grande decaida)}
Surgimento de todas as outras racas
e o termino da influencia dos deuses no mundo a baixa da magia
\chapter{6 era}
Queda dos imperios principalmente dos anoes
\chapter{7 era}
\chapter{8 era}
\chapter{9 era}
\chapter{10 era (modernidade de kuara)}
A dita era da renovacao 

A era da renovacao o resurgimento e espalhamento da presenca dos deuses e da magia uma dita 
renacensa dos clericados, paladinismo e monges uma reaproximacao do mistico e do misterioso.
Com o final da 5 era 

\end{document}
