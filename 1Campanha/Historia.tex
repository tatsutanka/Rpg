\documentclass{book}
\usepackage{blindtext}
\usepackage[T1]{fontenc}
\usepackage[utf8]{inputenc}
\usepackage{booktabs}
\title{O inicio do primeiro giro}
\author{Lucas Tatsuya Tanaka}
\date{\today}

\begin{document}

\maketitle
\tableofcontents

 \chapter{apresentacao}

 \chapter{Pre comeco de campanha}
 \section{cidade}
- Cidade murada (fortificacao media pra baixa)
- Nao apresenta uma nobreza 
- Cidade apresentava uma alta burguesia porem apos a contrucao de novas rotas muitos se mudaram da cidade abandonando a cidade e levando todos os beneficios que trazia juntos com eles os que acabaram ficando hoje acabando vivendo de aparencias pelo contato com o lider da cidade como se fosem nobrez 
- a capital praticamente nao se importa com o que ocorre dentro da cidade com excessao dos impostos des de que os impostos sejam devidamente cobrados os lideres da cidade podem fazer o que bem entenderem 
- 

fazia parte da rota principal de comercio a 10 a 20  anos atras porem apos  a construcao de novas rotas o dinheiro parou de chgar na cidade e agora maiores impostos sao cobrados de maneiro quase ditatorialfazendo

\chapter{Inicio}

- **nota 1:** fazer as pessoas se encontarem juntos devido a uma divida com uma pessoa 
- **nota 2** Apresentar as pessoas uma sociedade muito dividida 
- **nota 3** As pessoas estao dentro de uma cidade media, que fica relativamente longe da capital

\chapter{Meio}

- 

\chapter{final}

- **Nota 1** quero que as pessoas cheguem pelo menos a chegar em contato com o lider da cidade que controla a guarda e gram impostos abusivos sobre a populacao    
\end{document} 

